\documentclass[12pt, letterpaper]{article}
\usepackage{graphicx}
\graphicspath{ {Images/} }

\begin{document}
\noindent \textbf{\large Assignment 3 Heuristic Search }\newline
\noindent \textbf{By Brandon Young and Ruicheng Wu}

\bigskip

\noindent \textbf{A. Interface}

Here are three examples of the GUI. The first image shows the overall look of the GUI. The buttons located on the top-left of the interface are used to interact with the GUI. Clicking on a cell on the grid shows its h, g and f values below the buttons and above the grid. 

\medskip

\includegraphics[scale=0.30]{"a-map1-0"}

\medskip 

Below is the control panel of the GUI. All of the I/O interactivity is located here. The red "Show Path" button is used to upload a path file to show a green path on the grid. The yellow "File Output" is used to output a file representing the current grid. The dark blue "Upload" button is used to upload and show a grid from an existing file. The light blue "Cell Info" button is for uploading a file with the f,g and h values for cells in the grid.

\medskip

\includegraphics[scale=0.4]{"control-panel"}

\medskip

This third image shows a closer look at the actual grid. Red squares represent blocked cells, gray for hard-to-traverse cells and white for unblocked cells. A black circle within a square shows that the cell has a highway on it. Green P's show the path. The start (in blue) and goal (in purple) are either represented by S and G respectively or pie-shaped symbols.

\medskip

\includegraphics[scale=0.40]{"a-map1-4"}


\pagebreak %----------------------------------------------------------------------


\noindent \textbf{B. UCS, A* and Weighted A* Implementations}

There are four examples shown below show the path from running uniform-cost search (UCS), A* and weighted A* respectively on the same map.

1. Uniform-cost search:\newline
\medskip
\textbf{Path length: 81.08, Nodes Expanded: 11595,	Run Time: 0.384272}\newline
\noindent \includegraphics[scale=0.21]{"b-map5-1-u"}\newline

2. A * search ($h=1$, Euclidean distance): \newline
\medskip
\textbf{Path length: 81.08, Nodes Expanded: 8155,	Run Time: 0.287288s}\newline
\noindent \includegraphics[scale=0.25]{"b-map5-1-a-real"}\newline

3. A* search ($h=4$, Euclidean squared): \newline
\medskip
\textbf{Path length: 137.33, Nodes Expanded: 234,	Run Time: 0.056771s}\newline
\noindent \includegraphics[scale=0.21]{"b-map5-1-a"}\newline

4. Weighted A * search ($h=4$, $w=2.0$):\newline
\medskip
\textbf{Path length: 144.03, Nodes Expanded: 107,	Run Time: 0.05289s}\newline
\noindent \includegraphics[scale=0.2]{"b-map5-1-w-2"}\newline


\pagebreak %--------------------------------------------------------------------

\noindent \textbf{C. Optimizations}

To optimize the search algorithms, we used a Python dictionary as a heap, rather than a Python list (array), to implement the closed list more efficiently. Since we just needed to insert or check if a pair of coordinates existed inside the close list, a heap would speed up the check and may help the insertion. Overall, checking if a element exists takes O(n) for an array and O(1) for a heap.

For the heap, we initially used the sum of coordinate values to use as the key. However, this meant for every (x, y) there is a matching pair (y, x) with the same key. In addition, as the sums increased, there were a greater number of pairs that summed to the same value. As a result, we decided to distinguish (x, y) by multiplying x with a hash code value. With some testing, higher hash codes lowered the average size of a list for each key so we ended up using 91 as the hash code to speed up search checks.

We also optimized the way the heuristic was applied to the grid. At first we applied the heuristic over the entire graph before running the search algorithm. However, this wasted resources when only a portion of the overall grid was searched. Instead, we compute the heuristic during the search, to ensure only relevant cells in the grid have the heuristic applied to them.

\pagebreak %----------------------------------------------------------------

\noindent \textbf{D. Heuristics}

\noindent 1. Euclidean Distance

Euclidean distance is also a admissible/consistent heuristic as it works on a subproblem of the grid problem where any degree of movement is allowed. We also cut the distance by 4 to mimic movement along a highway, since the best possible case is if a highway leads from start to finish.

The formula for cell $(x_1, y_1)$ and goal $(x_2, y_2)$ is:
$$h((x_1,y_1)) = \frac{\sqrt{(x_2-x_1)^2 + (y_2-y_1)^2}}{4}$$

\bigskip
\noindent 2. Manhattan Distance

This heuristic assumes every cell in the grid is an unblocked cell with a highway on it. Then it computes the shortest path using only horizontal and vertical steps. We assumed every cell is a highway because using only unblocked cells results in higher than usual h-values. Often a large portion of the optimal path is along a highway, so the heuristic should reflect this.

The formula for cell $(x_1, y_1)$ and goal $(x_2, y_2)$ is:
$$h((x_1, y_1)) = 0.25 * (dx + dy)$$
where $dx = |x_2 - x_1|$ and $dy = |y_2 - y_1|$. The $0.25$ comes from the cost it takes to move horizontally or vertically in one step.

The heuristic is inadmissible for the grid problem because diagonal movements are normally allowed. As a result, Manhattan distance slightly overestimates the actual cost to the goal. Nevertheless, this metric can come in handy for sequential A* search because of that pessimism.

\bigskip
\noindent3. Diagonal Distance Heuristic

The best admissible/consistent heuristic we used was the diagonal distance heuristic. The diagonal distance heuristic is similar to the Manhattan distance heuristic, except diagonal steps are allowed. 

The formula for cell $(x1,y1)$ and goal $(x2, y2)$ is:
$$h((x1, y1)) = 0.25 * (dx + dy) +  (\frac{\sqrt{2}}{4} - 2(0.25)) * min(dx, dy)$$
where $dx = |x2 - x1|$ and $dy = |y2 - y1|$. The 0.25 comes from the cost to move horizontally and vertically between highway cells and $\frac{\sqrt{2}}{4}$ is the cost from moving diagonally between highway cells.

Diagonal distance is better than Manhattan distance because it is admissible and Manhattan is not. Diagonal distance was preferred over Euclidean distance because the movements it allows is closer to the movements allowed on the actual grid. Therefore, the heuristic value will come closer to the actual optimal distance than the value from Euclidean distance.

\bigskip
\noindent 4. Euclidean Squared

The formula for cell $(x_1, y_1)$ and goal $(x_2, y_2)$ is:
$$h((x_1,y_1)) = \frac{(x_2-x_1)^2 + (y_2-y_1)^2}{16}$$

This heuristic is similar to Euclidean distance, except it squares the output from Euclidean distance. As a result, it output very high values and it is not admissible. However, because the square root operation is not used, some computation time is saved. In general, this heuristic is pretty bad because its h-values will tend to be much higher than the g-values, but that may come in handy with sequential A* search.

\bigskip
\noindent 5. Sample Heuristic

The heuristic given in the assignment instructions. Not admissible because it does not consider highway movement.

The formula for cell $(x_1, y_1)$ and goal $(x_2, y_2)$ is:
$$h((x_1,y_1)) = \sqrt{2} * min(dx, dy) + max(dx, dy) - min(dx, dy)$$
where $dx = |x_2 - x_1|$ and $dy = |y_2 - y_1|$. 

\pagebreak %---------------------------------------------------------------

\noindent \textbf{E. UCS, A*, Weighted A* Statistics}

For the graphs below, h1 = Euclidean distance, h2 = Manhattan distance, h3 = diagonal distance, h4 = Euclidean squared, h5 = sample heuristic. Also, UCS = uniform-cost search, a = A* search, wa = weighted A* search and w = weight used for weighted A*

\medskip

\noindent \includegraphics[scale=0.65]{"avg-nodeexpanded"}

\medskip

\noindent \includegraphics[scale=0.65]{"avg-pathlength"}

\medskip

The time graph below was recorded in seconds.

\medskip

\noindent \includegraphics[scale=0.65]{"avg-runtime"}

\medskip

The memory graph below was recording the maximum size of the fringe running the algorithm

\medskip

\noindent \includegraphics[scale=0.65]{"avg-mem"}

\medskip

\pagebreak %---------------------------------------------------------------

\noindent \textbf{F. Analysis of UCS, A* \& Weighted A*}

At first glance, uniform-cost search (UCS) finds the shortest path, but it has the highest number of nodes expanded and time taken by far, compared to the other methods. This provides evidence how UCS sacrifices speed and memory for the guarantee of finding an optimal path.

In comparison, A* (heuristic search using the diagonal distance or euclidean distance heuristics),  tended to run much quicker than UCS, while still finding the shortest path. A* also expanded roughly a third less nodes than UCS. When using inadmissible heuristics, the heuristic search ran very quickly and expanded very few nodes (like with heuristic 4, Euclidean squared), but at the expense of longer paths.

Weighted A* search is similar to heuristic search with the Euclidean squared heuristic in that its runtime was quicker than A*, as expected, but the path length suffers slightly. With the additional weight factor bias the search, the heuristic can search more greedily with a good heuristic, but may return a sub-optimal path. Just a slightly higher weight like 1.5 (weight = 1 is the same as A*) reduces the time and number of nodes expanded significantly, while the average path length increases slightly. With a larger weight, like 2.5, the effects of weighted A* search are even more noticeable. A higher weight leads to even fewer nodes expanded and an even shorter run time with only a slight increase from the optimal path. In general, for the "good" heuristics (heuristics 1, 2 and 3), the sacrifice is not as severe as for other heuristics (heuristics 4 and 5). \newline

Between the heuristics themselves, heuristics 1 (Euclidean distance), 2 (Manhattan distance) and 3 (diagonal distance) performed the best in terms of path length. Heuristic 4 (Euclidean squared) and heuristic 5 (sample heuristic) performed the best in terms of time and nodes expanded.

The heuristic results were mostly expected, with diagonal distance performing slightly better than Euclidean distance. In heuristic search with no weights, diagonal distance and Euclidean distance both found the optimal path. Surprisingly, Manhattan distance's average path length comes really close to the optimal path length and, at the same time, it performs better in terms of nodes expanded and time in comparison to diagonal and Euclidean distance. With higher weights, Manhattan distance resulted in a noticeably worse path than the admissible heuristics with the same weights. 

Euclidean squared and the sample heuristic perform the worst in terms of path length by almost a factor of 2. It should be noted that the sample heuristic did not consider highways, which resulted in its inadmissibility. Moreover, with normal heuristic search, the sample heuristic's path comes fairly close to the optimal one. On the other hand, the Euclidean squared heuristic's path is consistently much longer than the optimal path, with or without weights. While these heuristic find longer paths, they also have huge saves in time and expand practically zero nodes relative to other searches.

Out of all of the heuristics, the sample heuristic is the only one whose average path length is significantly affected with weights 1.5 and 2.5. Other heuristics only have a slight change in average path length with the same weights.


In conclusion, Euclidean distance and diagonal distance are the best two heuristics to use among all of the five proposed ones. Diagonal distance outperforms Euclidean distance in nearly every aspect, except with weights involved, Euclidean distance's average path length fairs a bit better. To save on time, either weight the heuristic slightly (w = 2.5) or use the Euclidean squared or sample heuristics. 

\pagebreak %---------------------------------------------------------------

\noindent \textbf{G. Sequential A* Implementation}

 The path below was found using sequential A* using w1 = 1.5 and w2 = 1.5.

Path length: $ 95.024 $

Time (number of nodes expanded): $ 11095 $

Time (In seconds):$ 0.653 $

\noindent \includegraphics[scale=0.2]{"g-sns-map3-1"}

\medskip
\pagebreak %---------------------------------------------------------------

\noindent \textbf{H. Sequential A* Analysis}

\noindent \includegraphics[scale=0.6]{"s-nodeexpanded"}

\medskip

\noindent \includegraphics[scale=0.6]{"s-pathlength"}

\medskip

\noindent \includegraphics[scale=0.6]{"s-runtime"}

\medskip

The memory graph below was recording the sum of maximum size of the fringe running each heuristic search algorithm

\medskip

\noindent \includegraphics[scale=0.6]{"s-mem"}

\medskip


analysis here(h3 is the one we chose to be anchor heuristic)

\medskip

\noindent We run 16 different w1,w2 pairs to collect data. Through our observation, Sequential A* search should be generally slower than regular A* search since it runs multiple heuristic search.But after we increased $w_1$ to 2.5 and keep $w_2$ as low as possible, the run time becomes shorter than A* search. 
\medskip

\noindent The number of node expanded is directly related to both $w_1,w_2$. Since A* is optimal, so it uses the least memory to store data in fringe while sequential A* search need several fringes for each heuristic search. Hence, normally the memory required is larger than A* search. 
\medskip

\noindent Both $w_1,w_2$ can effect the run time performance, but it is believed that $w_1$ has more control on it. Also we observe $ w_1=2.5 $ has an averagely best performance on every aspect. Although both $ w_1 $ and $w_2$ can affect the performance,but with twisting the value of $w_2$,we can have a better run time performance while other stats are same,but this does not mean by infinite increasing $ w_2 $,we will get better run time performance,less node expanded and keep the path length in a not so dramatic increasing. Checkout example with $ (w_1=2.5,w_2=1.5;w_1=2.5,w_2=2.5;w_1=2.5,w_2=10) $, it has the worst suboptimal path. So this means there is a non-ignorable trade off between increasing $w_2$ and maintaining a decent path length.
\medskip

\noindent The best weight variable is $ w_1=2.5,w_2=2.5 $

\medskip



\pagebreak %---------------------------------------------------------------

\noindent \textbf{I. Sequential A* Proof}

\noindent \textbf{Part A:}

\noindent At first we assume $ key(s,0)= w_1*h_0(s)+g_0(s) > w_1 * g^*(s_{goal}) $

\medskip

Then we assume a least path from $s_{start} $ to $s_{goal} $ as P = $ \pi(s_0=s_{start},...,s_k=s_{goal}) $. Then we pick a state $ s_i $ that is not unexpanded by the anchor search.It should also be in the open set at index 0,which means $ s_i \subset OPENSET_0 $.
It is always possible to find such state since $ s_{start} $ is inserted at the initialization.Once any consecutive state $ s_j $ is expanded doing the anchor search, $ s_{j+1} $ is guaranteed to be inserted in the $ OPEN_0 $ set. 

In the meantime, $ s_k=s_{goal} $ won't be expanded in the anchor search.The pseudocode states that whenever $ s_{goal} $ have the MinKey in the $ OPEN_0 $, it will terminate and return the path.

Concerning of $ g_0(s_i) $. When i = 0,
$$ g_0(s_i) = g_0(s_{start})=0 \leq w_1 * g*(s_i)$$
because $g*(s_{start})=g_0(s_{start})=0$.

If $i \neq 0$, we can certainly know $ s_{i-1} $ has already been expanded in the anchor search. When chose the $ s_{i-1} $ to expand, we would have $g_0(s_{i-1}) \leq w_1 * g^*(s_{i-1})$ from the given property.

We can conclude(s` as successor of s)

$$g_0(s_i) \leq g_0(s_{i-1})+ c(s_{i-1},s_i)$$
$$\leq w_1 * (g^*(s_{i-1})+ c(s_{i-1},s_i))$$
$$=w_1*g^*(s_i)$$
because $ s_{i-1},s_i \subset optimal path$

So we now have $g_0(s_i) \leq w_1*g^*(s_i)$. We apply this equation,

$$key(s_i,0)= w_1*h_0(s_i)+g_0(s_i) \leq w_1 * g^*(s_i) +  w_1 * h_0(s_i)$$
$$ \leq w_1 * g^*(s_i) +  w_1 *c*(s_i,s_{goal})$$

$h_0$ is our anchor search heuristic, so it is consistent and admissible,thus,

=$  w_1*g^*(s_{goal}) $

Since $ s_i \subset OPEN_0$ and $key(s_i,0) \leq w_1*g^*(s_{goal}) < key(s,0)$, we have a contradiction to our initial assumption. Hence it proves  $ key(s,0) \leq w_1 * g^*(s_{goal}) $.

\bigskip

\noindent \textbf{Part B:}

\noindent We first clarify that there are 3 cases the sequential A* search will terminate, find path with anchor search, or by other inadmissible search or no solution.
 
\medskip

\noindent If it is terminated with anchor search finding the path, with the given property, we will obtain,

$$g_0(s_{goal}) \leq w_1 * g^*(s_{goal})$$
$$\leq w_1 * w_2 * g^*(s_{goal})$$ since both $ w_1 $ and $ w_2 $ are greater than 1.

\medskip

\noindent if it is terminated with other search, then

$$g_i(s_{goal}) \leq w_2 * OPEN_0.Minkey()$$
$$\leq w_2*w_1*g^*(s_{goal})$$

So we have the final path cost to be within $ w_1*w_2 $ factor of the optimal cost.

If we encountered the 3rd case, from part A we just proved, we know $ OPEN_0.Minkey() \leq w_1 * g^*(s_{goal}) $. $OPEN_0=\emptyset$ if and only if $OPEN_0.Minkey() = \infty ==> g*(s_{goal}) \geq \infty$,and we could still say it is bound within $ w_1*w_2 $ although there is no finite solution. 



\end{document}